\section{Analyzing the Microcode Engine}

The first step, when verifying a design, is to create a verification plan. 
It contains all functionalities of the design, which have to be checked.
Additionally, completion criteria are defined. They indicate that the verification is complete, when all of them are met.
Following the verification plan of the MCE is introduced.

\subsection{Instruction Set}

The instruction set is the most complex part of the verification plan, since every instruction has to be verified individually.
First of all, it has to be checked, that only valid opcodes are used, as invalid ones cannot be associated with an operation and can result in malfunctions
within the MCE. Additionally all possible values for any field has to be covered for each individual instruction.\\
Also combinations of similar fields have to be covered, so the data paths are examined.
Meaning, that all combinations of \emph{RD}, \emph{RS1} and \emph{RS2} have to be included.
Furthermore, each valid resulting condition code has to be written into each register of the CCR and it has to be checked that each instruction produces correct
results.\\
In addition to these items targeting individual instructions, the interaction between instructions has to be covered.
Most important is, that the transition between opcodes is examined, so the MCE can handle all combinations of instructions.
But furthermore, the transition between the same kind of fields between instructions is covered.
Meaning, that for example the destination address of an instruction does not influence the destination address of the following one.

\subsection{Start Interface}

For the start interface has to be covered, that the MCE was started using each possible start address within the instruction memory.
Furthermore, it has to be checked, that the MCE is actually started, when the \emph{start} signal is asserted and also started at the specified address.
Finally, it has to be checked, that the MCE is not started, when \emph{start} is deasserted.

\subsection{Instruction Memory Interface}

Within the instruction memory interface, it has to be covered, that each address of the instruction memory can be written. 
The data does not have to be covered specifically, since this is already done by collecting the coverage for the instructions, as mentioned above.
It has to be checked furthermore, that data is only written into the instruction memory, when \emph{write\_en} is asserted.

\subsection{External Condition Code Interface}

The only thing covered within the external condition code interface is, that each condition code bit has been asserted or deasserted, respectively and that it
can influence the execution of the BCC instruction.
Since all combination of bits are allowed, no checks have to be performed. 

\subsection{Call Stack Overflow Interface}

For the call stack overflow interface has to be covered, that an overflow can actually occur.
Furthermore, has to be checked, that the signal is deasserted after reset.
Additionally, it has to be checked, that it is asserted, whenever a call stack overflow occurs and also deasserted, whenever the MCE is started.
In all other cases, the signal has to be constant.

\subsection{Done Interface}

Similar to the call stack overflow interface, it has to be covered that the MCE can actually be stopped.
Furthermore, has to be checked, that the signal is deasserted after reset.
Additionally, it has to be checked, that it is asserted, whenever a STOP instruction is executed and also deasserted, whenever the MCE is started.
In all other cases, the signal has to be constant.

\subsection{Memory Interface}

The memory interface is the most complex interface of the MCE. Multiple checks have to be performed to ensure, that only valid transactions occur on the
interface. Following these checks are described.

\subsubsection{Address Checks}

When an transaction has been started, the address has to remain stable until \emph{access\_complete} is asserted.

\subsubsection{Write\_en Checks}

The \emph{write\_en} signal is asserted only for a single clock cycle. It cannot be asserted, while an access is pending on the memory interface.

\subsubsection{Write\_data Checks}

When an transaction has been started, the write data has to stay stable until \emph{access\_complete} is asserted.

\subsubsection{Read\_en Checks}

The \emph{read\_en} signal is asserted only for a single clock cycle. It cannot be asserted, while an access is pending on the memory interface.

\subsubsection{Read\_data Checks}

No specific requirements are laid down on the \emph{read\_data} signal.

\subsubsection{Access\_complete Checks}

The \emph{access\_complete} signal is only asserted, when an access is pending on the memory interface.
It is asserted for a single clock cycle.

\subsubsection{Invalid\_address Checks}

The \emph{invalid\_address} signal can only be asserted simultaneously with \emph{access\_complete}.
It is also asserted for a single clock cycle.

\subsection{Internal Components}

Since only little information about the internal state of the MCE can be gathered via its interfaces, additional data has to be collected from some of the
internal components of the engine. 
Thereby malfunctions within the engine can be detected, as soon as the happen.
Additionally these information can be used to check the execution of an instruction right after it has been completed.

\subsubsection{Program Counter}

It has to be covered, that the PC can be each address of the instruction memory. Furthermore the transition between these addresses has to be covered.
It needs to be checked, that the PC is initialized, when the MCE is started and always updated after an instruction has been fetched.

\subsubsection{Call Stack}

Addresses have to be pushed to as well as popped from the call stack. However, no address can be popped from an empty call stack.
Whenever an address is pushed to the call stack, its size increments and whenever one is popped from it, its size decrements.
Additionally, the PC has to be restored, when an address is popped from the stack.
The call stack should be able to have each valid size and an overflow has to occur, when an address is pushed to a full stack.

\subsubsection{Scratchpad}

It has to be checked, that data is read from or written to the register specified by the corresponding address signal of a port, when its enable signal is
asserted.
Additionally has to be covered, that each port is accessed, each address of the scratchpad is used and the combination of both.
Meaning, that each address is used on each port.\\
Furthermore, for each individual port has to be covered, that each transition of addresses occurs.


\subsubsection{Condition Code Register}

For the condition code register has to be checked, that condition code is written into the selected register, when \emph{cc\_valid} is asserted.
It also has to be covered, that each register is written as well as each valid condition code.
Furthermore, each combination of both has to be covered. Meaning, that each valid condition code is written into each register of the CCR.
Finally, the transition between those has to be covered.