% das Papierformat zuerst
%\documentclass[a4paper, 11pt]{article}

% deutsche Silbentrennung
%\usepackage[ngerman]{babel}

% wegen deutschen Umlauten
%\usepackage[ansinew]{inputenc}

% hier beginnt das Dokument
%\begin{document}


\thispagestyle{empty}

% \begin{figure}[t]
%  \centering
%  \includegraphics[width=0.4\textwidth]{abb/hd_logo_small_sw_16cm_rgb}
% \end{figure}

%\begin{figure}[t]
% \centering
% \includegraphics[width=0.3\textwidth]{abb/hd_logo_small_sw_16cm_rgb}
%~~~~~~~~~~
% \includegraphics[width=0.3\textwidth]{abb/ziti-logo_150}
%\end{figure}


\begin{verbatim}


\end{verbatim}

\begin{center}
\Large{Heidelberg University}\\
%\Large{- Campus <Name> -}\\
\end{center}


\begin{center}
\Large{Institute of Computer Engineering (ZITI)}
\end{center}
\begin{verbatim}




\end{verbatim}
\begin{center}
\doublespacing
\textbf{\LARGE{Functional Verification of a Microcode Engine using the Universal Verification Methodology}}\\
\medskip
\textbf{\large{Master Thesis in Computer Engineering}}\\
\singlespacing
\begin{verbatim}

\end{verbatim}
%\textbf{Master of Science}\\
%\textbf{Computer Engineering (Technische Informatik)}\\
%\textbf{MScTI}
\end{center}
\begin{verbatim}

\end{verbatim}
\begin{center}

\end{center}
%\begin{verbatim}

% \end{verbatim}
% \begin{center}
% \textbf{zur Erlangung des akademischen Grades \\ Bachelor / Master of Science}
% \end{center}
% \begin{verbatim}






%\end{verbatim}
\vfil

\begin{flushleft}
\begin{tabular}{llll}
%\textbf{Topic:} & & Universal Verification Methodology Multi-Language & \\
& & \\
\textbf{Author:} & & Sebastian Wittka BEng. &
\\
& & \\
\textbf{Version of:} & & Mannheim, \today &\\
& & \\
\textbf{Referee:} & & Prof. Dr.-Ing. Ulrich Brüning &\\
\textbf{Co-Referee:} & & JProf. Dr. Holger Fröning &\\
\textbf{Supervisor:} & & Dr. Niels Burkhardt &\\
\end{tabular}
\end{flushleft}