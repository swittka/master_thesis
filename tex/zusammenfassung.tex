\section*{Zusammenfassung}

Diese Masterarbeit beschäftigt sich mit der funktionalen Verifikation einer Microcode-Engine, die vom Lehrstuhl für Rechnerarchitektur am Institut
für Technische Informatik der Universität Heidelberg entwickelt wurde. Sie erörtert, wie die Universal Verification Methodology in Kombination mit Metric Driven
Verification benutzt werden kann, um das korrekte Verhalten der Funktionseinheit sicher zu stellen.\\
Die in dieser Arbeit entwickelte Testbench zielt darauf ab, zufällige Stimuli mittels mehrerer Tests spontan zu erzeugen.
Diese Tests wurden entwickelt, um spezielle Funktionalitäten der Funktionseinheit zu behandeln, aber dennoch generell genug zu sein, um alle interessanten
Fälle zu erreichen.
Die generierten Stimuli werden dann verwendet, um das Design unter Test anzusteuern.\\
Zusätzlich wird ein auf Transaktionsebene basierendes Referenzmodell benutzt, um das Verhalten der Microcode-Engine vorherzusagen.
Des Weiteren wird der interne Status der Microcode-Engine zusätzlich zu ihren Schnittstellen überwacht, damit das Model noch enger mit dem Verhalten der
Funktionseinheit verbunden ist. Beide, die gesammelten und die vorhergesagten Ergebnisse werden dann verglichen, um eine sich selbst überprüfende Umgebung zu
erstellen. Schließlich werden Abdeckungsdaten gesammelt, um die Effektivität der Verifikation zu messen.